\section{Related Work}
In data center related work under traditional IP network, there are quite a few
existing work focusing on reducing redundant computations via caching, like
DryadInc \cite{Isard:2007:DDD:1272996.1273005}, Coment
\cite{He:2010:CBS:1807128.1807139}, the stateful bulk processing system
\cite{Logothetis:2010:SBP:1807128.1807138} and Nectar
\cite{gunda2010nectar}. And among them we choose Nectar as our main reference
and comparison target since it attempts to provide a more comprehensive solution
to the problem of automatic management of data and computation in data
center. In order to make direct and straightforward comparison, we borrowed the
data center use cases as incremental computing and sub-computation directly from
Nectar, attempting to prove that NDN solution could provide better solution than
Nectar, which to us, by somehow represents the existing most centralized special
server based solution under traditional IP network.  In NDN network, or even
content centric network as a wider scope, currently no existing working
published aiming at such solution for data center network scenario. The most
related work, if we just consider the appearance data center network scenario in
CCN related research, is in \cite{lee2010greening} where the author use data
center as one CCN use case to illustrate the energy saving performance of
CCN. Obviously, from the motivation to the evaluation, our work is totally
different from their work.  Currently the main concern of the community is still
in some fundamental problems such as general naming mechanism, data security,
routing, rather than such quite specific problem in specific scenario as data
center here. Good part of this is that our work could be relatively fresh in
ideas, but the negative part is that since plenty of more fundamental problem
existing to address, as well as the potential and future of NDN is unknown, our
work which focus on the quite particular data center intermediate result sharing
problem is like a building built on the unstable ground. Therefore the
contribution and value of our work quite largely depends on the success of
NDN. However, even through NDN were finally proved as unsuitable for the general
whole internet, for the individual data center, which is relatively small in
scale and centralized controlled by each companies or organizations, it still
could be promising if we could prove our NDN solution with attractive
improvement of performance compared with traditional IP – special servers based
solution. Since previous examples like optical circuit switches are getting more
popular and accepted, we do have the reason to believe that our work could have
quite much contribution if the solution could be enough promising, even if NDN
lost the bigger campaign.  About content centric network naming, which is one of
the key issues we want to address under the scenario, there’s quite a few
related works talking about the structures and rules of naming, such as
\cite{ghodsi2011naming} or \cite{primes}, but their main motivation is for
security or scalability under the scenario of the whole internet, which is very
different with ours.  There are also some existing studies on CCN caching. For
example, in \cite{carofiglio2011modeling}, the author developed an analytical
model for the performance evaluation of content transfer in CCN that allows
explicit characterization of steady state dynamics. Our work could draw
inspiration from this work in the network topology design part and therefore
design a better topology and caching strategy.
