\section{Refinement}
We discuss three refinements for our proposed design.
\subsection{Merge Intermediate Results}\label{sec:merge}
As described in Section \ref{sec:utilize_cache}, computation servers maintain a list
of intermediate results they have computed.  Therefore, a computation server is
able to merge these intermediate results by merging their names.  A data
requester can send an {\sc interest} with the merged name, whose {\sc data}
packet has not been cached.  The {\sc interest} will eventually reach the
computation server, who can fetch these unmerged intermediate results from cache
and combine them into a merged intermediate results.  If some parts of
intermediate results have no longer existed in cache, the computation server can
re-compute them.  Once a merged name replace unmerged ones, these unmerged
intermediate results will never be requested any more, and they will be ejected
from cache when their TTL expires or their spaces must be cleaned to cache new
data.  By merging intermediate results, we can always keep cache entries at a
small scale.
\subsection{Robustness \& Load Balance} \label{sec:robust}
An computation server will become an single point failure if it is the only one
server assigned for one operation.  Besides this, if the operation is
computation-intensive or very popular among applications, the computation server
will be overloaded easily.  Therefore, it is desired to assign multiple
computation servers for one operation.  To achieve that, all computation servers
should announce the same prefix for the operation they are assigned.  

Given an {\sc interest}, which computation server will receive the {\sc
  interest} is up to NDN routers' decision.  As a result, a computation server
may not know the computation results of another computation server performing
the same operation.  This would affect the correctness of metadata fetching
described in Section \ref{sec:utilize_cache}.  To avoid this problem, we use
SYNC, a NDN communication protocol which can synchronize name sets of
multi-parties, to synchronize name sets of intermediate results of different
computation server for one operation. 
\subsection{Garbage Collection} \label{sec:gg}
It is very important for our NDN-based solution to remove useless intermediate
results from cache while keeping the most popular intermediate results in cache,
in order to save storage resources and increase effectiveness of cache.  In NDN,
the maximum time for a {\sc data} packet to stay in cache is determined before
it is sent out from its producer, and will not change in the network.  While it
is easy to let useless {\sc data} get expired in NDN network, it is difficult to
make a popular {\sc data} stay in the cache of NDN routers.  Therefore, in order
to avoid re-computation, a computation server can store its popular intermediate
results locally in case they get expired in NDN routers.  The popularity of an
intermediate result can be evaluated based on the statistics of metadata query
described in Section \ref{sec:utilize_cache}.

Storage resources of computation servers are also limited. For those long-term
popular intermediate results, especially those merged huge-bulk of intermediate
results, it is not feasible to store them at computation servers.  Therefore, we
use Repo, a NDN storage technique which provides persistent storage, to store
this data.
