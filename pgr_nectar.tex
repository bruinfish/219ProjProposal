\section{Nectar's solutions}

This section briefly describes the Nectar's solutions. Nectar names every piece of intermediate
result by the Robin fingerprints of the sub-program and the input
data that generated it.  The intermediate results are cached
in a distributed file system and managed by a centralize cache
service while all the access to the intermediate results must go through
this cache service.  Then, Nectar adopts a rewrite-based approach at
the compiler level to hook the program with the intermediate results.
Before a program is submitted to the computation cluster, a
client-side program rewriter will identify the intermediate data that
will be useful to this program and rewrite the program in a way that when the program is being executed, it will
retrieve these intermediate results from the cache storage and merge
them with the new results. Also, the rewriter, based on the frequencies of
the inquiring of each sub-program's result, decides whether to cache
sub-programs' results.  Finally, because all the intermediate results
are managed by a centralized service, Nectar simply implements a
cost-benefit algorithm to garbage collect the obsolete intermediate
results.

While we agree with that Nectar is going toward a right direction to reduce the
computation cost in data centers, we found that Nectar's rewriter-based
solution imposes many unnecessary constraints to the uses of intermediate results.
First, in order for rewriter to rewrite the programs, the
programs must be written in the certain language that is
understandable to the rewriter. In fact, Nectar can only support the programs that were
written in the C# while many programs ran in data centers do not
adopt C# or Microsoft's DryadLINQ. Also, as mentioned in the Nectar's paper,
even if a program is written in the C#, if it invokes any external library written in
other languages, these parts of computations cannot benefit
from the intermediate results because rewriter cannot understand them. 

We argue that, a more flexible design is needed, so that such a idea
of using intermediate can be widely applicable, and different programs written in the different
languages can all benefit from the intermediate results. 

Second, in Nectar, because rewriting is performed at the compilation time, the
flexibility of using intermediate results is largely limited. The
following example demonstrate this problem: Consider a program
contains four steps of sub-computations: {p1, p2, p3, p4}. With Nectar, this program can only
benefit from the intermediate results that were generated by the
Prefix-sub-programs i.e. {p1, p2, p3, p4}, {p1, p2, p3}, {p1, p2} or
{p1}, because when Nectar rewrites the program, it doesn't know the
input data of the computation p2, p3 or p4 and so cannot identify the
intermediate results that were generated by the sub-programs that
start with these computations. Also, rewrite-based solution can only
modify the programs based on the situation during the compilation time. If
there comes other new intermediate data that could benefit the program after the
compilation time, these data cannot be used in the programs. We argue
that, with our design, which is more flexible, we can better utilize the
potential benefit of intermediate results for the above-mentioned
cases.

Finally, Nectar also suffers the static granularity problems both in
the program decomposition and the intermediate data cache. First, there is no space for
program developers to decide how their programs are decomposed. Some
programs need finner-grained decomposition because each small step of
it might be useful to other programs. Meanwhile, some programs might
not need such finner-grained decomposition. But with Nectar, all the programs
are decomposed in the same way, which could either waste the
storage, caching unnecessary intermediate results by too
finner-grained decomposition, or fail to utilize
the potential of intermediate results by too coarse-grained
decomposition. 

On the other hand, in Nectar, with static granularity of intermediate
result cache, two piece of intermediate results
overlapping in the input data range cannot be merged.
For example, two pieces of intermediate results: one is generated by
input data ranging from 0 to 100, another one is generated by input
data ranging from 90 to 190. A program can choose either of them to
use, which loses considerable benefit of intermediate results.

In this project, we consider designing a NDN-based distributed
computing system that can utilize the intermediate results. The
reasons of adopt NDN are two-fold: first, in NDN, every piece of data
inherently has a name and is transmitted in the network by this name.
Therefore, with NDN, we don't need a complicated and inflexible rewriter and a centralized cache
service to hardwire programs with intermediate results. We can solve the
problems more naturally and elegantly. Second, there are some good
features of NDN we can benefit from. For example, NDN allow the router
to cache the data, reducing the network traffic. Therefore, by careful design, we
can not only make a system using intermediate results become more
flexible, but also more efficient than a IP-based solution, such as
Nectar.

