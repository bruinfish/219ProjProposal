\documentclass[journal]{IEEEtran}

\hyphenation{op-tical net-works semi-conduc-tor}


\begin{document}

\title{Distribtued Computation over NDN-based Data Center Network}

\author{Zhiyang Wang,
        Cheng-Kang Hsieh,
        and Yingdi Yu}

\maketitle

\begin{abstract}
The abstract goes here.
\end{abstract}

\begin{IEEEkeywords}
Data Center Network, NDN, Distributed Computation
\end{IEEEkeywords}

\IEEEpeerreviewmaketitle



\section{Introduction}
\IEEEPARstart{D}{ata}-intensive distributed computing has been a major challenge
in data centers for a long time.  Without well-managed data and computation, a
large portion of computations in a data center have to be unnecessarily
repeated.  In order to avoid redundant computation, Pradeep, {\it et al.}
proposed Nectar \cite{gunda2010nectar}, a system in which intermediate computation
results can be cached so that they can be reused by following computation.
Evaluation results suggested that caching intermediate results can save at most
99\% redundant computations in some cases \cite{gunda2010nectar}.

However, there are still several unresolved issues in caching intermediate
computation results in current data center network.  First, a centralized cache
server, as proposed by Nectar, may become a hot spot when computing
significantly relies on cached intermediate results.  Therefore, a distributed
cache system appears to be a more potential solution.  Second, when several
computing servers need to re-use the same cached intermediate result, they have
to fetch the cached data individually from the cache server.  Such a data
communication mode may work in some small-volume data-exchanging and highly
distributed system (e.g. DNS), but may become inefficient in data center
network where data exchanged are much more bigger and computing servers are
relatively close to each other in terms of network topology.  

NDN 


\section{Problem Statement}
Use caching in data center.
\subsection{Naming Intermediate Result}
\subsection{Network Topology}
\subsection{Comparison With IP-based Solution}
\section{Related Work}
\section{Timeline}

\bibliographystyle{abbrv}
\bibliography{proposal}

%\begin{thebibliography}{1}
%\bibitem{2010osdiGunda}
%Nectar
%\end{thebibliography}
\end{document}


