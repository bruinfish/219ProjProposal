\documentclass[journal]{IEEEtran}

\hyphenation{op-tical net-works semi-conduc-tor}


\begin{document}

\title{Distribtued Computation over NDN-based Data Center Network}

\author{Zhiyang Wang,
        Cheng-Kang Hsieh,
        and Yingdi Yu}

\maketitle

\begin{abstract}
The abstract goes here.
\end{abstract}

\begin{IEEEkeywords}
Data Center Network, NDN, Distributed Computation
\end{IEEEkeywords}

\IEEEpeerreviewmaketitle



\section{Introduction}
\IEEEPARstart{D}{ata}-intensive distributed computing has been a major challenge
in data centers for a long time.  Without well-managed data and computation, a
large portion of computations in a data center have to be unnecessarily
repeated.  In order to avoid redundant computation, Pradeep, {\it et al.}
proposed Nectar \cite{gunda2010nectar}, a system in which intermediate computation
results can be cached so that they can be reused by following computation.
Evaluation results suggested that caching intermediate results can save at most
99\% redundant computations in some cases \cite{gunda2010nectar}.

However, there are still several unresolved issues in caching intermediate
computation results in current data center network.  First, a centralized cache
server, as proposed by Nectar, may become a hot spot when computing
significantly relies on cached intermediate results.  Second, when several
computing servers need to re-use the same cached intermediate result, they have
to fetch the cached data individually from the cache server.  Such a data
communication mode may work in some small-volume data-exchanging and highly
distributed system (e.g. DNS), but may become inefficient in data center network
where data exchanged are much more bigger and computing servers are relatively
close to each other in terms of network topology.

Named-Data Network \cite{jacobson2009networking}, a novel network architecture
which can cache data inside networks instead of servers, appears to be a
potential solution to the problems mentioned above.  In NDN, data are named and
fetch by their names, rather than through a end-to-end connection between a
requester and a data provider.  With a specific name, a data packet can be
cached along the path (more specifically speaking, NDN routers) it has traveled
through.  When some other requesters ask for the same data, the request can be
satisfied by data cached in NDN routers without reaching the corresponding data
provider.  In this way, NDN intrinsically provides a distributed caching
service, and avoids end-to-end connections. 

Although NDN has many attractive features, it is not very feasible to provide
intermediate computation results caching service in NDN.  There are two open
questions: 1) how to name intermediate computation results to maximize the
caching efficiency and 2) what is the best network topology for NDN-based data
center network.  These two issues will be discussed in detail in Section
\ref{sec:problem_statement}.

In this project, we plan to answer the two questions above by building a
prototype of NDN-based data center network.  In order to evaluate the
performance of the prototype, we will provide two specific applications
supported by the prototype: 1) simple SQL query in a distributed database and 2)
distributed word occurance analysis.  As we will describe in Section
\ref{sec:problem_statement}, each application represents a different type of
distributed data computing: incremental computing and sub-computing
respectively.  We will also compare the performance of our prototype with the
existing IP-based solution, Nectar, and try to derive some general conclusions
which can be used for future data center network designs.

\section{Problem Statement}\label{sec:problem_statement}
To make the problem discussion more clear, we first describe two applications
that will be implemented over our prototype. And then two related issues will be
discussed in detail.

The first application is a simple SQL query in a distributed database.  In this
database, data are stored on multiple servers in a distributed way, and queries
are processed in a MapReduce style.  When a query is issued, related entries
will be grouped by mapers, and then further processed by reducers.  When new
entries have been injected into the database, all related entries have to be
re-grouped if intermediate results caching is not used.  When the grouped
entries can be cached as intermediate results, mapers only need to group
recently injected entries which can be combined with cached results for reducer
to process.  Such a computing mode is called as incremental computing.  

The second application is a distributed word occurance analysis.  Unlike the
previous application, data set is persistent in this application.  In order to
get the final results, the computation can be divided into several steps.  Some
steps may require the computation results of some other steps.  And the same
computation results may be re-used by more than one computing servers.  Such a
computing mode is called as sub-computing.

\subsection{Name Intermediate Results}
We use the first application to discuss the naming issue in NDN-based network.
In traditional IP network, a maper will transfer data to a reducer, while in
NDN, a reducer is notified a task, and then fetches data needed for the task.
The reducer issues an {\sc Interest} message which contains the name of
requested data.  The name should be able to represent the timeliness of the
task.  For example, if a reducer needs all the grouped entries before a
timestamp $t_i$, and there are several pieces of entries injected at timestamps
$t_{i-3}$, $t_{i-2}$, $t_{i-1}$ before $t_i$, then all intermediate grouped
entries, if cached, should be able to be matched with a name containing the
timestamp $t_i$.  However, the sequential nature of timestamp conflicts with the
hierarchical structure of NDN's naming mechanism which determines caching
efficiency to a large extent. Therefore, we have to find a compatible naming
mechanism for incremental computing.

\subsection{Network Topology}
DCN over NDN should have a specific-designed topology. Current DCNs usually adopt
a hierarchical topology, such as fat-tree, to achieve scalability, fault tolerance,
and high capacity. Such a design assumes that the communication is end-to-end-based, 
but this is not the case for NDN. In NDN, a data requester is not necessary to 
retrieve data from the data producer but can retrieve the data from a router that 
have cached the data. If the network is based on a hierarchical topology, most traffic
will go through the core routers. For NDN, this means that the core routers should provide
larger cache to achieve higher cache hit rate. But, increasing the cache size also increases
the time for NDN routers to perform the longest-prefix matching routing procedure and makes
these upper-level routers become the bottleneck of the communications.

\subsection{Comparison With IP-based Solution}
The contribution and value of our work largely rely on whether our NDN 
based solution could actually provide better performance than the 
traditional IP-based centralized server solution, although 
conceptually the data-centric NDN network should be suitable for such 
data-centric use case in data center. In the current stage, there are 
a few advantages we assume: First, as decentralized, our solution 
could be better in robustness and scalability. Second, as most of data 
are cached in local router, the latency will be shorter for data 
fetching since in some cases where two servers are under the same 
router or neighbor routers. Third, the network topology and 
infrastructure is simpler since there’s no extra special server which 
all outer servers must connect to. By leveraging its job toward 
routers who by nature will do caching in NDN network, the whole 
solution could be less costly. But since there’s no existing real well 
accepted NDN product in the market, this point is hard to verify. 
However, as centralized, the centralized special server has all the 
data from every server, and therefore could enable large-scale 
aggregation or analysis algorithm. We need to verify whether it is 
important under our scenario and whether our NDN solution could also 
provide such function in other efficient ways. 
Moreover, the advantages and disadvantages in this project could also 
provide an insight to the more general comparison between these two 
solutions in other scenario, such as data collection in participate 
sensing or other data center related topics such as leverage the 
master’s task toward decentralized routers. 

\subsection{Related Work}
In data center related work under traditional IP network, there are 
quite a few existing work focusing on reducing redundant computations 
via caching, like DryadInc [1], Coment [2], the stateful bulk 
processing system [3] and Nectar [4]. And among them we choose Nectar 
as our main reference and comparison target since it attempts to 
provide a more comprehensive solution to the problem of automatic 
management of data and computation in data center. In order to make 
direct and straightforward comparison, we borrowed the data center use 
cases as incremental computing and sub-computation directly from 
Nectar, attempting to prove that NDN solution could provide better 
solution than Nectar, which to us, by somehow represents the existing 
most centralized special server based solution under traditional IP 
network. 
In NDN network, or even content centric network as a wider scope, 
currently no existing working published aiming at such solution for 
data center network scenario. The most related work, if we just 
consider the appearance data center network scenario in CCN related 
research, is in [5] where the author use data center as one CCN use 
case to illustrate the energy saving performance of CCN. Obviously, 
from the motivation to the evaluation, our work is totally different 
from their work. 
Currently the main concern of the community is still in some 
fundamental problems such as general naming mechanism, data security, 
routing, rather than such quite specific problem in specific scenario 
as data center here. Good part of this is that our work could be 
relatively fresh in ideas, but the negative part is that since plenty 
of more fundamental problem existing to address, as well as the 
potential and future of NDN is unknown, our work which focus on the 
quite particular data center intermediate result sharing problem is 
like a building built on the unstable ground. Therefore the 
contribution and value of our work quite largely depends on the 
success of NDN. However, even through NDN were finally proved as 
unsuitable for the general whole internet, for the individual data 
center, which is relatively small in scale and centralized controlled 
by each companies or organizations, it still could be promising if we 
could prove our NDN solution with attractive improvement of 
performance compared with traditional IP – special servers based 
solution. Since previous examples like optical circuit switches are 
getting more popular and accepted, we do have the reason to believe 
that our work could have quite much contribution if the solution could 
be enough promising, even if NDN lost the bigger campaign. 
About content centric network naming, which is one of the key issues 
we want to address under the scenario, there’s quite a few related 
works talking about the structures and rules of naming, such as [6] or 
[7], but their main motivation is for security or scalability under 
the scenario of the whole internet, which is very different with 
ours. 
There are also some existing studies on CCN caching. For example, in 
[8], the author developed an analytical model for the performance 
evaluation of content transfer in CCN that allows explicit 
characterization of steady state dynamics. Our work could draw 
inspiration from this work in the network topology design part and 
therefore design a better topology and caching strategy. 

\section{Timeline}
Week 4-6: Paper review and system & algorithm design 
Week 7:  System & algorithm formulation 
Week 8: Experiments design, System & algorithm implementation and 
evaluation 
Week 9: Final presentation and report preparation 


\bibliographystyle{abbrv}
\bibliography{proposal}

%\begin{thebibliography}{1}
%\bibitem{2010osdiGunda}
%Nectar
%\end{thebibliography}
\end{document}


